% [[- Chapter title
\chapter*{Introduction}
% -]] Chapter title

% [[- Horace's Epodes
\section*{Horace's \textit{Epodes}}
% -]] Horace's Epodes

% [[- Epode 13
\section*{\textit{Epode} 13}
% -]] Epode 13

% [[- Meter
\section*{Meter}

The meter of \textit{Epode} 13 is the second Archilochian. This meter uses
couplets made up of (1) a dactylic hexameter followed by (2) a line combining
an iambic dimeter and a hemiepes.  An iambic dimeter is two iambic feet, but
one iambic foot requires \textit{two} iambs in Greek and Latin meters.  (A
Latin iambic dimeter is thus eight syllables, whereas an Shakespearean iambic
pentameter is only ten.)  The initial syllable of each foot in an iambic meter
is anceps: that is, it may be long or short.  Since there is a mandatory word
end at the end of the dimeter, the final syllable is also anceps.  A hemiepes
is is two and a half dactylic feet.  (The name comes from Greek: `hemiepes'
means ``half an epic verse".  A hemiepes is roughly half of a dactylic
hexameter, or a single line of epic poetry.)  In the second Archilochian, there
are no substitutions of dactyls for spondees within the hemiepes.  The final
syllable of the hemiepes is also anceps since it stands at the end of the
verse.  The pause at the end of every line allows poets to use either a long or
a short syllable as if it were a long.\newline

\indent\metra{\m\mbb\m\mbb\m\c\mbb\m\mbb\m\bb\m\mb}

\indent\indent\metra{\mb\m\b\m\mb\m\b\mb\cc\m\bb\m\bb\mb}\newline

The meter suits the poem very well.  The hexameter would suggest epic, i.e.
heroic, themes.  Iambic meters, on the other hand, were more suitable to
lighter or lower genres and themes.  \textit{Epode} 13 suggests a background of
war and uses Achilles as an \textit{exemplum}, but it is also a \textit{carpe
diem} poem and the Achilles story involves non-Homeric elements.
% -]] Meter

% [[- About The Text
\section*{About The Text}

I made the text for this edition by comparing the editions of \citet{sb1985}
and \citet{mankin1995}. In almost all cases, what I print comes from one of
those two editions. However, I've gone my own way a few times---mostly over
punctuation.

The apparatus is minimal and follows the style used in several recent volumes
in the \textit{Cambridge Greek and Latin Classics} series. In every case, the
notes are keyed to the word or phrase in question after which I indicate
where an alternative comes from in the simplest possible way.

The key below explains how the apparatus presents this information.

\begin{description}%
    [style=sameline,leftmargin=70pt,labelwidth=\widthof{\textbf{Name}}]
    \item[m] one or more manuscripts
    \item[mss] the consensus opinion of most or all of the manuscripts
    \item[Name] a conjecture suggested by the named author
\end{description}
% -]] About The Text
