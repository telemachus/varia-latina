% [[- Chapter title
\chapter*{Introduction}
% -]] Chapter title

% [[- Horace's Epodes
\section*{Horace's \textit{Epodes} and \textit{Epode} 13}

The \textit{Epodes} are a collection of seventeen highly varied poems from early in Horace's poetic career.  Although there are no certain dates for his early writings, the \textit{Epodes} are likely to have been his third published collection of poems.  Around 35 BCE Horace published his first book of \textit{Sermones} (\textit{Conversations}), usually called \textit{Satires} in English.  Then around 30 BCE Horace published a second book of \textit{Sermones} and a book that he called his \textit{Iambi}, or \textit{Iambics}.  In later times, the title \textit{Epodes} displaced \textit{Iambi}.  It is unclear whether the second collection of \textit{Sermones} or the \textit{Iambi} was published first, but in any case the \textit{Epodes} cannot have been published before 31 BCE.  \textit{Epode} 9 refers to the battle of Actium, which took place on September 2\textsuperscript{nd} 31 BCE.

The historical background for the \textit{Epodes} is the final stages of the drawn out civil war that followed the assassination of Julius Caesar.  The men who took Caesar's life first faced off against Caesar's supporters and heirs.  After the Caesarians, led by Anthony and Octavian, defeated the so-called Liberators, they turned on each other. In the end, Octavian and his forces defeated Anthony and Cleopatra in the battle of Actium.  A number of \textit{Epodes} refer to the political uncertainties and wars of this period in the 30s.  It is very likely that poem 13 does as well.

From a literary point of view, Horace took Greek iambic poetry as his primary model.  He particularly followed the work of Archilochus, who was active in the mid-600s BCE.  But Horace no doubt read a great deal of other Greek iambic poetry, particularly that of Hipponax (circa 540 BCE) and Callimachus (circa 310-240 BCE).  The surviving remains of Greek iambic poetry is very fragmentary, but some key characteristics are clear.  Iambic poetry was often very aggressive, it could be quite crude and vulgar, it frequently involved exemplary stories involving animals (e.g. the fox and the grapes), and poets often spoke in the mouth of imagined characters rather than in their own voice (as in lyric or elegy).

Several of Horace's \textit{Epodes} share many of these features from Greek iambic poetry, but others are much closer to lyric.  \textit{Epode} 13 is one of the more lyric poems.  In setting, style, and message, it resembles one of the \textit{carpe diem} poems from Horace's \textit{Odes}.  Horace and a friend enjoy the pleasures of wine and song while a storm rages outside.  Horace uses the occasion to suggest larger lessons about human life and mortality.

% -]] Horace's Epodes

% [[- Meter
\section*{Meter}

The meter of \textit{Epode} 13 is the second Archilochian. This meter uses couplets made up of (1) a dactylic hexameter followed by (2) a line combining an iambic dimeter and a hemiepes.  An iambic dimeter is two iambic feet, but one iambic foot requires \textit{two} iambs in Greek and Latin meters.  (A Latin iambic dimeter is thus eight syllables, whereas an Shakespearean iambic pentameter is only ten.)  The initial syllable of each foot in an iambic meter is anceps: that is, it may be long or short.  Since there is a mandatory word end at the end of the dimeter, the final syllable is also anceps.  A hemiepes is is two and a half dactylic feet.  (The name comes from Greek: `hemiepes' means ``half an epic verse".  A hemiepes is roughly half of a dactylic hexameter, or a single line of epic poetry.)  In the second Archilochian, there are no substitutions of dactyls for spondees within the hemiepes.  The final syllable of the hemiepes is also anceps since it stands at the end of the verse.  The pause at the end of every line allows poets to use either a long or a short syllable as if it were a long.\newline

\indent\metra{\m\mbb\m\mbb\m\c\mbb\m\mbb\m\bb\m\mb}

\indent\indent\metra{\mb\m\b\m\mb\m\b\mb\cc\m\bb\m\bb\mb}\newline

The meter suits the poem very well.  The hexameter would suggest epic, i.e.  heroic, themes.  Iambic meters, on the other hand, were more suitable to lighter or lower genres and themes.  \textit{Epode} 13 suggests a background of war and uses Achilles as an \textit{exemplum}, but it is also a \textit{carpe diem} poem and the Achilles story involves non-Homeric elements.

% -]] Meter

% [[- About The Text
\section*{About The Text}

I made the text for this edition by comparing the editions of \citet{sb1985} and \citet{mankin1995}. In almost all cases, what I print comes from one of those two editions. However, I've gone my own way a few times---mostly over punctuation.

Below the text is an \textit{apparatus criticus} that gives information about variants in the text.  The apparatus is minimal and follows the style used in several recent volumes in the \textit{Cambridge Greek and Latin Classics} series.  I use what is called a `positive' apparatus.  Every note begins with the reading I adopt and where it comes from.  After that I list alternatives of interest.  

The key below explains how the apparatus presents this information.

\begin{description}%
    [style=sameline,leftmargin=70pt,labelwidth=\widthof{\textbf{Name}}]
    \item[m] one or more manuscripts
    \item[mss] the consensus opinion of most or all of the manuscripts
    \item[Name] a conjecture suggested by the named author
\end{description}

For further discussion of textual issues, see the chapter on Textual Notes later in the book.

% -]] About The Text
