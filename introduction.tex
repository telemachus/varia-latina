% [[- Chapter title
\chapter*{Introduction}
% -]] Chapter title

% [[- Ovid's Tristia
\section*{Ovid's \textit{Tristia}}
% -]] Ovid's Tristia

% [[- Style and meter
\section*{Style and meter}
% -]] Style and meter

% [[- About the text
\section*{About The Text}

I made the text for this edition by comparing the editions of \citet{owen1915}
and \citet{hall1998}. In almost all cases, what I print comes from one of those
two editions. However, I've gone my own way a few times---mostly over
punctuation.

The apparatus is minimal and follows the style used in several recent volumes
in the \textit{Cambridge Greek and Latin Classics} series. In every case, the
notes are keyed to the word or phrase in question after which I indicate
where an alternative comes from in the simplest possible way.

The key below explains how the apparatus presents this information.

\begin{description}%
    [style=sameline,leftmargin=70pt,labelwidth=\widthof{\textbf{Name}}]
    \item[m] one or more manuscripts
    \item[mss] the consensus opinion of most or all of the manuscripts
    \item[Name] a conjecture suggested by the named author
\end{description}
% -]] About the text
