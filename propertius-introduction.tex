% [[- Chapter title
\chapter*{Introduction}
% -]] Chapter title

% [[- Horace's life
\section*{Propertius' Life}

TODO

% -]] Horace's life

% [[- Horace's Carmina and the two spring poems

% -]] Horace's Carmina and the two spring poems

% [[- Meter
\section*{Meter}

The meter of Propertius' elegies, like all Roman elegy, is the elegiac couplet. This meter has a long history, stretching back to 7\super{th} century Greece.

TODO

\indent\metra{\m\mbb\m\mbb\m\mbb\m\m\b\b\m\b\b\m\mb}

\indent\indent\metra{\m\mbb\m\mbb\m\c\m\b\b\m\b\b\mb}\newline
% -]] Meter

% [[- About The Text
\section*{About The Text}

I made the text for this edition by comparing the editions of \citet{goold1990}. Everything in my text comes from one of those two editions. I also drew alternative readings from their notes.

Below the text is an \textit{apparatus criticus} that gives information about difficulties in the text.  The apparatus is minimal and follows the style used in several recent volumes in the \textit{Cambridge Greek and Latin Classics} series.  I use what is called a `positive' apparatus.  Every note begins with the reading I adopt and where it comes from.  After that I list alternatives of interest.  

The key below explains how the apparatus presents this information.

\begin{description}%
    [style=sameline,leftmargin=70pt,labelwidth=\widthof{\textbf{Name}}]
    \item[m] one or more manuscripts
    \item[M] the consensus opinion of most or all of the manuscripts
\end{description}

For further discussion of textual issues, see the chapter on Textual Notes later in the book.

% -]] About The Text
