% [[- Chapter title
\chapter*{Tristia}
\def\ind{%
    \hspace{2em}%
}
% -]] Chapter title

% [[- Hypenation help
\hyphenation{}
% -]] Hypenation help

% [[- Text
\beginnumbering
\autopar
\firstlinenum{5}

\setstanzaindents{3,0,1}
\setcounter{stanzaindentsrepetition}{2}
\stanza
Ille ego qui \var{fuerim}{fueram}{m}, tenerorum lusor amorum,&
quem legis, ut noris, accipe posteritas.&
Sulmo mihi patria est, gelidis uberrimus undis,&
milia qui novies distat ab urbe decem.&
editus \var{hic}{hinc}{m} ego sum, nec non, ut tempora noris,&
cum cecidit fato consul uterque pari.&
si quid id est, usque a proavis vetus ordinis heres,&
non modo fortunae munere factus eques.&
nec stirps prima fui: genito sum fratre creatus,&
qui tribus ante quater mensibus ortus erat.&
Lucifer amborum natalibus affuit idem:&
una celebrata est per duo liba dies;&
haec est armiferae festis de quinque Minervae,&
quae fieri pugna prima cruenta solet.&
protinus excolimur teneri, curaque parentis&
imus ad insignes urbis ab arte viros.&
frater ad eloquium viridi tendebat ab aevo,&
fortia verbosi natus ad arma fori:&
at mihi iam puero caelestia sacra placebant,&
inque suum furtim Musa trahebat opus.&
saepe pater dixit: `studium quid inutile temptas?&
Maeonides nullas ipse reliquit opes.'&
motus eram dictis, totoque Helicone relicto&
scribere \var{temptabam}{conabar}{m} verba soluta modis:&
sponte sua carmen numeros veniebat ad aptos,&
et quod \var{temptabam}{conabar}{m} \var{dicere}{scribere}{m} versus erat.&
interea tacito passu labentibus annis&
liberior fratri sumpta mihique toga est,&
induiturque umeris cum lato purpura clavo:&
at studium nobis, quod fuit ante, manet.&
iamque decem vitae frater geminaverat annos&
cum perit, et coepi parte carere mei.&
cepimus et tenerae primos aetatis honores,&
eque viris quondam pars tribus una fui.&
curia restabat: clavi mensura coacta est:&
maius erat nostris viribus illud onus.&
nec patiens corpus, nec mens fuit apta labori,&
sollicitaeque fugax ambitionis eram,&
et petere Aoniae suadebant tuta sorores&
otia, iudicio semper amata meo.&
temporis illius colui fovique poetas,&
quotque aderant vates, rebar adesse deos.&
saepe suas volucres legit mihi grandior aevo,&
quaeque \var{nocet}{necet}{m} serpens, quae iuvet herba, Macer.&
saepe suos solitus recitare Propertius ignes,&
iure sodalicii qui mihi iunctus erat.&
Ponticus heroo, Bassus quoque clarus iambis&
dulcia convictus membra fuere mei.&
et tenuit nostras numerosus Horatius aures,&
dum ferit Ausonia carmina culta lyra.&
Vergilium vidi tantum, nec avara Tibullo&
tempus amicitiae fata dedere meae.&
successor fuit hic tibi, Galle, Propertius illi;&
quartus ab his serie temporis ipse fui.&
utque ego maiores, sic me coluere minores,&
notaque non tarde facta Thalia mea est.&
carmina cum primum populo iuvenilia legi,&
barba resecta mihi bisve semelve fuit.&
moverat ingenium totam cantata per urbem&
nomine non vero dicta Corinna mihi.&
multa quidem scripsi, sed quae vitiosa putavi,&
emendaturis ignibus ipse dedi.&
\var{tum}{tunc}{m} quoque, cum fugerem, quaedam placitura cremavi,&
iratus studio carminibusque meis.&
molle Cupidineis nec inexpugnabile telis&
cor mihi, quodque levis causa moveret, erat.&
cum tamen hic essem, minimoque accenderer igni,&
nomine sub nostro fabula nulla fuit.&
paene mihi puero nec digna nec utilis uxor&
est data, quae tempus \vvar{perbreve}{Heinsius}{per breve}{mss} nupta fuit.&
illi successit, quamvis sine crimine coniunx,&
non tamen in nostro firma futura toro.&
ultima, quae mecum seros permansit in annos,&
sustinuit coniunx exulis esse viri.&
filia me mea bis prima fecunda iuventa,&
sed non ex uno coniuge, fecit avum.&
et iam conplerat genitor sua fata, novemque&
addiderat lustris altera lustra novem.&
non aliter flevi, quam me fleturus adempto&
ille fuit; matri proxima iusta tuli.&
felices ambo tempestiveque \vvar{sepulti}{mss}{sepultos}{Heinsius},&
ante diem poenae quod periere meae!&
me quoque felicem, quod non viventibus illis&
sum miser, et de me quod doluere nihil!&
si tamen extinctis aliquid nisi nomina restat,&
et gracilis structos effugit umbra rogos,&
fama, parentales, si vos mea contigit, umbrae,&
et sunt in Stygio crimina nostra foro,&
scite, precor, causam (nec vos mihi fallere fas est)&
errorem iussae, non scelus, esse fugae.&
Manibus hoc satis est: ad vos, studiosa, revertor,&
pectora, qui vitae quaeritis acta meae.&
iam mihi canities pulsis melioribus annis&
venerat, antiquas miscueratque comas,&
postque meos ortus Pisaea vinctus oliva&
abstulerat deciens praemia victor equus,&
cum maris Euxini positos ad laeva Tomitas&
quaerere me laesi principis ira iubet.&
causa meae cunctis nimium quoque nota ruinae&
indicio non est testificanda meo.&
quid referam comitumque nefas famulosque nocentes?&
ipsa multa tuli non leviora fuga.&
indignata malis mens est succumbere, seque&
praestitit invictam viribus usa suis;&
oblitusque \vvar{mei}{mss}{aevi}{Hall} ductaeque per otia vitae&
insolita cepi temporis arma manu;&
totque tuli terra casus pelagoque quot inter&
occultum stellae conspicuumque polum.&
tacta mihi tandem longis erroribus acto&
iuncta pharetratis Sarmatis ora Getis.&
hic ego, finitimis quamvis circumsoner armis,&
tristia, quo possum, carmine fata levo.&
quod \vvar{quamvis}{mss}{quamquam}{Hall} nemo est, cuius referatur ad aures,&
sic tamen absumo decipioque diem.&
ergo quod vivo durisque laboribus obsto,&
nec me sollicitae taedia lucis habent,&
gratia, Musa, tibi: nam tu solacia praebes,&
tu curae requies, tu medicina \var{mali}{venis}{m}.&
tu dux et comes es, tu nos abducis ab Histro,&
in medioque mihi das Helicone locum;&
tu mihi, quod rarum est, vivo sublime dedisti&
nomen, ab exequiis quod dare fama solet.&
nec, qui detrectat praesentia, Livor iniquo&
ullum de nostris dente momordit opus.&
nam tulerint magnos cum saecula nostra poetas,&
non fuit ingenio fama maligna meo;&
cumque ego praeponam multos mihi, non minor illis&
dicor, et in toto plurimus orbe legor.&
si quid habent igitur vatum praesagia veri,&
protinus ut moriar, non ero, terra, tuus.&
sive favore tuli, sive \var{hanc ego carmine}{aequo numine}{Hall} famam,&
iure tibi grates, candide lector, ago.\&

\endnumbering
% -]] Text
