% [[- Chapter title
\chapter*{Commentary}
% -]] Chapter title

% [[- Arrival of spring
\section*{Arrival of spring 1--8}

Horace begins by describing the arrival of spring in two stanzas.  In the first he shows the changes in weather and the effect of those changes on the human world.  In the second stanza he depicts spring's effect at the divine level.

\comment{1}{4}

The first stanza is a single, carefully-constructed sentence.  Horace uses an elaborate ascending tricolon to organize the opening: \textbf{(1) Solvitur \lips (2) trahuntque \lips (3) ac (a) neque \lips gaudet \lips (b) nec \lips albicant}.  In addition, the stanza is shaped around a chiasmus: the first and fourth lines focus on the weather and the natural world; the second and third depict human actions and reactions to the change in weather. 

\comment{1}{1}

\lem{Solvitur}: When the weather grows warmer, the snow and ice melts;\footnote{\textit{O} I 9.5 \textit{dis\textbf{solve} frigus}} this in turn frees the earth.\footnote{This poem line 10: \textit{terrae \lips solutae}}  Horace fuses and shifts these two ideas by saying that the winter is melted.\indent\lem{acris}: Ennius and Lucretius also apply \textit{acer, acris, acre - sharp, piercing, penetrating} to \textit{hiems}. See also Horace's own \textit{gelu \lips acuto} (\textit{O} I 9.3--4).\indent\lem{veris et Favoni}: Nearly a hendiadys: \textit{spring which the west wind announces}.  (The genitives are perhaps subjective with the verbal idea in \textit{vice}.)\indent\lem{Favoni}: Greeks and Romans often say that spring begins in early February when the west wind begins to blow again.\footnote{Pliny the Elder \textit{Naturalis Historia} (hereafter \textit{NH}) 2.122 and 18.337}  Roman poets often use the Greek name for the west wind, \textit{Zephyrus}, rather than the native Latin \textit{Favonius}.  This is the first of several distinctly Roman touches to the poem.

\comment{2}{2}

\lem{machinae}: Horace has in mind winches that would help roll drydocked ships back into the sea when the sailing season began.

\comment{3}{5}

\lem{iam}: Horace's repetition of \textit{iam} here may echo a spring poem of Catullus that includes notable repetition of the word.  \textit{iam} appears on lines 1, 2, 7, and 8 of Catullus 46.

\comment{3}{3}

\lem{aut arator igni}: Supply \textit{gaudet} from the previous clause.  \textit{igni} is ablative, not dative.  The verb \textit{gaudere} regularly takes an ablative of what leads the subject to rejoice.

\comment{4}{4}

\lem{albicant}: The verb \textit{albicāre} is a poetic and rare alternative to the more prosaic \textit{albēre}.

\comment{5}{5}

\lem{Cytherea}: Cythera, an island in the Aegean sea, was one of the places Venus was supposed to have been born.\indent\lem{Venus} is particularly associated with the arrival and power of spring.\footnote{Lucretius \textit{De rerum natura} (hereafter \textit{DRN}) I 10--16 and V 737--740}

\comment{6}{6}

\lem{decentes}: The participle of \textit{decēre} does not appear as an adjective before Horace, and he may have been the first to use it this way.  (Given how much poetry is lost, however, it is impossible to be certain.)  The adjective becomes common in later poets and prose writers as a high-style and elegant word, combining the ideas of beauty and appropriateness.

\comment{7}{8}

\lem{dum gravis \lips officinas}: Horace switches from images of beauty and dancing to Vulcan working underground in the ``shops of the Cyclopes".  The transition is associative: first, Vulcan is the husband of Venus; second, lightning storms were common in the spring.  Thus, Vulcan and the Cyclopes had a lot of work to do since they supplied Jupiter with his supply of lightning.\footnote{Lucretius \textit{DRN} VI 357--358 connects spring and lightning.}

\comment{7}{7}

\lem{Cyclopum}: In Homer's \textit{Odyssey} the Cyclopes are shepherds, but as early as Hesiod's \textit{Theogony} (circa 7th century BCE), they are also represented as blacksmiths who forge lightning for Zeus.\footnote{\textit{Theogony} 139--141}

\comment{8}{8}

\lem{visit} is from \textit{viso, visere} not \textit{video, videre}.\indent\lem{officinas} is a prosaic word, and this makes the transition from the dancing Nymphs and Graces to the working Cyclopes that much more emphatic.
% -]] Arrival of spring

% [[- Time to sacrifice

\section*{The need to sacrifice 9--12}

After describing the natural and divine world and the changes they are undergoing in the opening two stanzas, Horace picks up here with his advice for us.  In response to the arrival of spring, we must sacrifice to Faunus.  Faunus was an Italian god that the Romans eventually associated with the Greek Pan.  Horace may have chosen him for local color, as with \textit{Favonius} above in line 1. Additionally there was an urban festival for Faunus in early February.  So there is a natural association with the timing of the poem's opening.

\comment{9}{12}

\lem{nunc decet \lips nunc \lips decet}: The anaphora of \textit{nunc} and the repetition of \textit{decet} insist on the need for a sacrifice at the right time.  In addition, \textit{decet} provides a characteristically Horatian association between the \textit{decentes} Graces and the sacrifice.\footnote{This in turn helps to explain the transition to the next stanza.  More on that below.}  There is also an association between \textit{terram quatiunt} which the Graces and nymphs dance on (7), and the flower-bearing \textit{terrae} here.  In Greco-Roman mythology and iconography, it is common to describe the earth or sea as rejoicing, literally or figuratively, in the presence of a divinity.  Flowers grow, animals rejoice, the world reacts in sympathy to the god.

\comment{10}{10}

\lem{terrae \lips solutae}: In line 1, winter was in the process of melting away.  Here, nine lines later, the earth is released already and flowers are in bloom.  \citet[66]{nh1989} point out that ``in Mediterranean countries flowers are associated with spring and not with summer", but the temporal relations have struck readers in different ways.  On the one hand, some people argue for a fairly detailed temporal progression in the poem from early February through later February.  They argue that you can see this progression by recalling specific festivals in Rome for the month, as with the festival of Faunus mentioned above.\footnote{See below for another example.}  On the other hand, some readers argue that as a lyric poet, Horace is not interested in a week-to-week calendar of events so much as he is in the larger mood and tone of a season.  Thus, they say that we shouldn't make too much of the contrast between the first stanza and this one.

\comment{11}{11}

\lem{immolare}: This verb is derived from the noun \textit{mola}, a couarsely ground grain mixed with salt that was scattered onto (hence \textit{im} from \textit{in}) sacrificial victims.  It is commonly transitive, taking an accusative of what is sacrificed and a dative of reference indicating the god for whom the sacrifice is performed.  Horace uses the verb without a direct object\footnote{Such a treatment of a transitive verb is often called an \textit{absolute} use of the verb.} to mean \textit{perform a sacrifice} rather than \textit{sacrifice <something>}.  In the next line, Horace shifts the normal construction even further.\indent\lem{lucis}: \citet[82]{mayer2012} points out that sacred groves are specific  to Roman religion---as opposed to  the blended Greco-Roman religion in much Roman poetry---and that such groves are particularly associated with Faunus.  Again, Horace has deliberately made this poem highly \textit{Roman}.
\comment{12}{12}

\lem{agna \lips haedo}: Both are ablatives of means with an understood \textit{immolare} from the previous clause.  Normally, these would be accusative direct objects of the verb, but Horace avoids the standard construction of the verb.
% -]] Time to sacrifice

% [[- Death is coming
\section*{Memento mori! 13--20}

The poem takes a very sharp turn: the speaker reminds us that everyone dies and that the time to enjoy life is brief.  The addressee of the poem appears for the first time in line 14.  Lucius Sestius joined the army of Brutus after Julius Caesar was killed.  He survived the civil war between Brutus and Cassius, on one side, and Marc Antony and Octavian, on the other.  Although he was initially proscribed, he was eventually pardoned and returned to Italy.  In 23 BCE Augustus resigned as consul mid-year and named Sestius consul for the rest of the year.\footnote{A consul elected or named to serve out part of the year when the elected consul was unable to finish the year was known as a \textit{consul suffectus} or \textit{suffect consul}.  According to ancient historians, Augustus picked Sestius precisely because he had been on the side of the Liberators and he remained openly pro-Brutus even after he was pardoned and returned to Italy.  By making Sestius suffect consul, Augustus demonstrated both his mercifulness and his power.  He was merciful towards his enemies after he had thoroughly defeated them and had nothing left to fear.}  Sestius plays nearly no role in the poem, and nothing known about his personality connects him to its subject.  The choice may be a way of dating the collection, which was published in 23 BCE, or it may be a nod towards the politically important moment when August stepped down as consul.  Augustus had been one of the two consuls every year from 30 down until 23 BCE.\footnote{After 23, Augustus was not consul again for nearly twenty years.  He was consul only two more times, in 5 and 2 BCE.}

\comment{13}{13}

\lem{pallida Mors}: Quintilian, a \nth{1} century CE teacher of rhetoric, uses this phrase as example of a specific type of metonymy that ``displays cause from effect".\footnote{\textit{Institutio Oratoria} 8.6.27: \textit{id quod efficit ex eo quod efficitur}.}  Horace calls death, which causes paleness, pale itself.\indent\lem{aequo pulsat pede}: Two kinds of equality are important. First, death comes to rich and poor (\textit{pauperum tabernas/regumque turris} 13--14) alike.  Second, the foot of death is an equally important counter-balance to the dancing feet of the Graces and nymphs from earlier in the poem.\footnote{\textit{aeterno terram quatiunt pede} 7}\indent\lem{pulsat}: Horace portrays Death as an impatient guest, kicking the door of a house in order to get the attention of the people inside.

\comment{14}{14}

\lem{o}: The use of this interjection to introduce a vocative makes the appeal more emotional.

\comment{15}{15}

\lem{vitae summa brevis}: It's impossible to know by form whether \textit{brevis} modifies \textit{vitae} or \textit{summa}.  In addition, the meaning will amount to the same thing whether we read it as \textit{the sum of a brief life} or \textit{the brief sum of life}.

\comment{16}{17}

\lem{nox \lips fabulaeque Manes \lips et domus exilis Plutonia}: A tricolon \textit{crescens} of subjects for \textit{premet}.  As is common, \textit{premet} is singular in agreement with \textit{nox}, the subject closest to the verb.

\comment{16}{16}

\lem{fabulaeque Manes}: Both words are nouns, but it is not uncommon in many languages to use a noun adjectively to modify another noun.\footnote{Examples in English: \textit{car door}, \textit{History Department}. In Latin, \citet[84]{mayer2012} compares \textit{atavis \lips regibus} from \textit{O} I 1.  That phrase literally means \textit{grandfather kings}, but could be reasonably translated as \textit{ancestral kings}.}  So the phrase as a whole seems to mean \textit{the fabled Manes} or \textit{the Manes celebrated in stories and legends}.

\comment{17}{17}

\lem{Plutonia}: It is characteristic of high style to use an adjective instead of a genitive.  \textit{Plutonian house} sounds much grander than \textit{Pluto's house}.  The name \textit{Pluto} is another Roman touch.\indent\lem{mearis} is a syncopated form of \textit{meaveris}.  The verb \textit{meare} is poetic diction in Horace's time, although it becomes more common in Latin prose in later writers.\footnote{This is a frequent occurence: words that are poetic for Vergil and Horace are picked up by imperial prose writers and become more common in later Latin.}

\comment{18}{18}

This line refers to the practice of choosing one person at a symposium to be in charge of drinking.  This person decided the ratio of wine to water, and so they controlled the tone of the party in a sense. 

\lem{regna} is a poetic plural.  The person in charge of drinking could be called the \textit{magister} or \textit{arbiter} \textit{bibendi} (or \textit{potandi}).  Horace uses \textit{regna vini} as a poetic way of expressing the same notion: that someone will have complete power over the intensity of drinking.\indent\lem{talis}: The selection of the person in charge was apparently left to luck.  The symposiasts would roll dice, and the winner would become \textit{magister bibendi}.

\comment{19}{20}

In these last two lines Horace highlights elements from the Greek side of Greco-Roman culture.  Horace uses the Greek form of Lycidas' name, and he alludes to ideas about sexuality that are more often found in Greek sources than Roman ones.  Lycidas is young currently, and as such he is attractive to older men such as Sestius.  But as Lycidas grows older, he will become an object of women's desires.  This change reminds us of the passage of time, and that in turn reminds us of the inevitability of death and the need to enjoy life while we can.

\lem{tenerum}: The adjective \textit{tener, tenera, tenerum} is typically used in Roman love poetry for the object of desire, the beloved.\indent\lem{Lycidan}: The form is a Greek accusative singular.  Roman law specifically outlawed sexual relations with under-age citizens, but there was no such ban concerning slaves.  Hence Greek names are common in situations like this one.
% -]] Death is coming
