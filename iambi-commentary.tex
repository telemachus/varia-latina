% [[- Chapter title
\chapter*{Commentary}
% -]] Chapter title

% [[- A winter storm 1-3
\section*{A winter storm 1--3}

The opening lines describe a terrible winter storm.  It is not clear where or who the speaker is, and there is no initial indication who the speaker is addressing.  As such, we are plunged into the storm ourselves as readers.

\comment{1}{1}

\lem{Horrida}: The adjective \textit{horridus, -a, -um} often means `causing horror, dreadful, horrible' (\textbf{OLD} 6), but it is also frequently used of rough or choppy seas (\textbf{OLD} 1c).  As such it may anticipate the description of the sea in the next two lines.  In any case, this word starts the poem on a grim note.\indent\lem{caelum contraxit}:  Neither word is unusual, but the combination is effective since it simultaneously creates several connotations.  The verb can mean `reduce in size, diminish, narrow' (\textbf{OLD} 2): the gathering clouds reduce the visible sky, but the phrase also suggests `gather into a pile, amass, muster' (\textbf{OLD} 4): the storm piles cloud upon cloud.\footnote{This is how Porphyrio, an ancient commentator on Horace, understood the phrase.  He carefully explains that we should understand \textit{caelum} to mean \textit{nubes} and the entire expression to mean `thicken'.}  In addition to these more literal meanings, `frontem contrahere' or `frontem contrahere' means `frown, scowl' (\textbf{OLD} 1b) and the verb alone can mean `sadden, depress' (\textbf{OLD} 3).  These meanings suggest the appearance of the people looking at the storm.  They also anticipate the metonymy of Jupiter for sky in the next line.

\comment{1}{2}

\lem{imbres \lips nivesque deducunt Iovem}: A violent and surprising phrase.  \textit{Jupiter} means \textit{sky} by metonymy.  This metonymy is common, but Horace reinvigorates the potential cliche through an inversion of normal phrasing.  In everyday language, the sky pours down rain and snow, but the poet imagines that ``rain and snow drag down the sky".  As \citet[215]{mankin1995} says, the image also suggests an assault on the normal order of Olympus.  That in turn was likely to make contemporary Roman readers think of their recent civil wars.  Such readers would have been able to read between the lines better than we can if the poem first circulated during a specific crisis in the late 30s.

\comment{2}{2}

\lem{nunc \lips nunc}: Insistent anaphora emphasizes the storm's force.\indent\lem{silvae}: The \textit{v} is treated as a vowel, and the word is scanned \begin{metrica}s\-il\-u\={ae}\end{metrica}.  (The technical term for this process is \textit{vocalic diaeresis}.)

\comment{3}{3}

\lem{Threico Aquilone} is scanned \begin{metrica}Thr\=e\-ic\-i\=o \-Aqu\-il\=on\-e\end{metrica}.  That is, the initial syllable of \textit{Threicio} is not a diphthong, and there is a hiatus between the two words.  Greek writers from Homer onward described the winds, especially the north wind, as living in or coming from Thrace.  Roman poets followed suit.

% -]] A winter storm 1-3

% [[- Let's drink while we're young 3-6
\section*{Let's drink while we're young 3--6}

The poem shifts from the storm outdoors to the party indoors, as Horace encourages a friend to drink with him and forget their troubles.\footnote{\textit{C} I 9 displays a structure very similar to this poem's.  The first stanza describes a winter storm outdoors, the second invites the addressee to drink, and the third begins \textit{permitte divis cetera}---which echoes \textit{cetera mitte loqui} below.}

\comment{3}{3}

\lem{amici} is vocative plural.  We have no precise sense of who these friends are, and Horace does not address this poem to any specific, named person.  This is unusual for a sympotic poem, but it fits the poem's overall reticence about its context.

\comment{4}{4}

\lem{de die}: The meaning of the preposition here is not entirely obvious.  With a period of time, \textit{de} often means \textit{starting with, at} (\textbf{OLD} 3); and \textit{de die} with a reference to drinking often refers to day-drinking, which Romans considered extravagant and potentially scandalous.  In addition, \textit{de} can indicate the person or thing from whom something is taken or received, as well as the person over whom a victory is won (\textbf{OLD} 6).  The primary meaning is probably \textit{from the day} (as source), but the suggestions of early drinking and victory over the bad weather are also likely present.\indent\lem{virent genua}: The reference to knees may surprise us initially, but it suits the poem's implicit military context.  In the \textit{Iliad} Homer often focuses on knees as a symbol of heroic strength.  The knees of a warrior or powerful animal carry them ``easily across a field" (\textit{Iliad} 6.511).  And when a hero dies, his ``knees are loosened" (i.e. their strength is undone).\footnote{For example \textit{Iliad} 5.176, ``he (Diomedes) loosened the knees of many noble men".}  As \citet[216--217]{mankin1995} notes, Horace's language echoes that of Theocritus, a Hellenistic Greek poet. Theocritus has one man tell another to join Ptolemy's army with the words: ``Do what is fitting while your knee is green" (\textit{Idylls} 14.70).  (For other Horatian uses of \textit{virere} to express youthful strength and vitality, compare \textit{I} 17.33, \textit{C} I 9.17 and I 25.17. See also \textit{OLD} 2.)

\comment{5}{5}

\lem{obducta solvatur fronte senectus}: Three spondees start this phrase, slowing the line down after the initial dactyl \textit{et decet}.  Meter mirrors sense as the line turns towards gloomy brows and the misery of old age.  \textit{obducta fronte} is ablative of separation with \textit{solvatur}.  The phrase means a `covered over' or `darkened' brow.  Since \textit{obducere} is often used of clouds or darkness (\textbf{OLD} \textit{obduco} 5, 6), these words recall the connotations of \textit{caelum contraxit} from line 1.  \textit{senectus} means literally `old age', but here it appears to mean `gloom' or `melancholy'.  This associated meaning is common with the Latin noun \textit{senium}, but would be unique for \textit{senectus}, as \citet[217]{mankin1995} notes.

\comment{6}{6}

Elaborate word order: \textit{tu} at the start of the line balances \textit{meo} at verse end. Between those words there is a play on a Golden Line in chiastic order.  A Golden Line is an entire verse made up of balanced pairs of nouns and adjectives that surround a centrally-placed verb.  The imperative \textit{move} is surrounded by a chiasmus of accusative (\textit{vina}) ablative (\textit{Torquato}) ablative (\textit{consule}) accusative (\textit{pressa}).\indent\lem{tu}: Horace turns from general advice to speak to one person in the group.  The pronoun serves to shift focus and draw attention to what follows. Horace probably refers here to a specific custom: at a Roman party, the guests suggest a specific type of wine that they want to drink \citep[136]{kilpatrick1970}.\indent\lem{vina Torquato \lips consule pressa meo}: Wine jars would be labelled with their date, so Horace is asking for a specific vintage.  Romans referred to the date of years in one of two ways: by the year's consuls or with a numerical date since Rome was founded (\textit{ab urbe condita}).  In 65 BCE, L. Manlius Torquatus was consul with L. Aurelius Cotta and Horace was born.  (Hence \textit{my} Torquatus.)\indent\lem{move} suggests that the addressee should remove the wine jar from the \textit{apothēca}, a storeroom where Romans kept wine.  (Compare C III 21.6: \textit{\textbf{moveri} digna bono die}, describing a wine that `deserves to be taken out on a good day'.)

% -]] Let's drink while we're young 3-6

% [[- Focus on current pleasures; leave the rest to the gods 7-10
\section*{Focus on current pleasures; leave the rest to the gods 7--10}

Don't worry about anything else. Perhaps a god will improve things, but now we should enjoy a symposium.

\comment{7}{7}

\lem{cetera}: Possibly a euphemism, modelled on a Greek usage.  In Greek it was idiomatic to say, e.g.,  ``Soon we will find out if the outcome is good or \textit{otherwise}" instead of ``good or \textit{bad}".  So here Horace urges his addressee not to speak about \textit{other things}, and those things are presumably bad.\footnote{If this interpretation is correct, then \textit{cetera} would implicitly balance \textit{benigna} at the end of the line.}  The speaker models his own advice: we never learn from this poem exactly what other issues hang over the men's heads.  We could view \textit{cetera} as merely a reference to the weather, but the poem seems too anxious and foreboding for that.\indent\lem{mitte loqui}: The use of \textit{mittere}, meaning `avoid; stop' (\textbf{OLD} 4b),  with an infinitive may be colloquial.  The usage shows up a number of times in Roman comedy.  The phrase serves as a poetic equivalent for the prosaic \textit{noli loqui}.\footnote{For the idea, compare \textit{C} I 9.9 \textit{permitte divis cetera}. For the idiom, compare \textit{C} I 38.3 \textit{mitte sectari}.}\indent\lem{deus}: In Homer, the narrator can always say precisely which god is responsible for various events because he possesses omniscience as a result of his connection to the Muses.  On the other hand, characters in the epics tend to say ``a god" or ``some god" instead of specifying a god since their knowledge of divine matters is minimal.  The vague \textit{deus} reinforces the poem's message that humans should know their limits and leave the rest to the gods.\indent\lem{fortasse}: The best we can hope for is `perhaps'.  Depending on what you make of the rest of the poem, the word might be hopeful or resigned or bitter.  This word also looks forward to the short life of Achilles, which is said to be certain (\textit{certo} 15).\footnote{Compare the words of Aeneas as he attempts to console his men in \textit{Aeneid} 1, after a storm has shipwrecked them: dabit deus his quoque finem. \lips forsan et haec olim meminisse iuvabit (1.199, 1.203).  ``A god will put an end to these things. \lips Perhaps someday it will be a pleasure to remember even these things."  Aeneas' vague \textit{haec} is like Horace's \textit{cetera}.}\indent\lem{benigna} is ablative with \textit{vice} on the next line and not neuter with \textit{haec}.

\comment{8}{8}

\lem{reducet in sedem}: The phrase suggests returning something to its proper, previous position or state.  As such, it picks up on the idea of a revolutionary overthrow implicit in line 2.  In addition, the phrase implies that Horace and Amicius may be abroad.  The verb \textit{reducere} can mean `to bring back home' (\textbf{OLD} 1b), and the noun \textit{sedes} can mean `home' (\textbf{OLD} 4).  This is another reason to hear the suggestion of a military campaign somewhere far from Rome.

\comment{8}{10}

The structure of this sentence may be difficult to see: \textit{nunc iuvat et perfundi et levare}.  \textit{et \lips et} join the two infinitives, each of which is a subject of \textit{iuvat}, used impersonally.  Translate \textit{It is pleasing both to \lips and to \lips}.

\comment{8}{8}

\lem{nunc} directs attention to the present moment, as opposed to the previous two lines which hint at future dangers and hope for future improvements.  At the same time, however, \textit{nunc} recalls \textit{nunc \lips nunc} in line 2 which focused on the terrible storm happening outside right now.\indent\lem{Achaemenio}: Achaemenes was the (possibly mythical) anscestor of the Achaemenid Persian kings: e.g. Cyrus, Xerxes, and Darius.  More importantly for Horace, \textit{Persian} was synonymous with luxurious.

\comment{8}{9}

\lem{Achaemenio \lips nardo}: Ablative of means with \textit{perfundi}.

\comment{9}{9}

\lem{perfundi}: When added to a verb the prefix \textit{per-} can suggest thoroughness : the action is carried through to completion.  In addition, it can intensify the force of a verb.  (See the \textbf{OLD} entry on \textit{per-}.)  Here it emphasizes the extravagance of the proposed celebration.\indent\lem{nardo \lips fide}: Perfume and music are expected at a Greco-Roman drinking party.  \textit{fide} is from \textit{fidēs, fidis, f.\ - lyre} not  \textit{fidēs, fidēī, f.\ - trust, faith}.\indent\lem{Cyllenea}:  Hermes, the inventor of the lyre, was born on Mt. Cyllene.  Hence, the lyre is Cyllenean.  The word is scanned \begin{metrica}\textit{C\=yll\=en\=e\=a}\end{metrica}.  This hexameter thus has a spondaic fifth foot.\footnote{According to \citet[220]{mankin1995} this is only one of four spondaic hexameters in Horace's lyric poetry.  All four involve proper names.}

\comment{9}{10}

\lem{fide \lips sollicitudinibus}: The ablative phrases are not parallel here.  \textit{fide Cyllenea} is ablative of means, and \textit{diris sollicitudinibus} is ablative of separation following \textit{levare pectora}.  (We might say ``Relieve your hearts of care.") 

% -]] Focus on current pleasures; leave the rest to the gods 7-10

% [[- Exemplum: Chiron's advice to Achilles 11-18
\section*{Exemplum: Chiron's advice to Achilles}

[Let's enjoy a symposium], just as Chiron recommended to Achilles before he went off to fight the Trojan war.

The final section of the poem is an \textit{exemplum}: a mythical or historical story that provides a positive or negative example for the audience's behavior.  Before the Trojan war, the wise centaur Chiron gave advice to Achilles that sounds very like the advice Horace just gave to his friend.  Although the parallels between Chiron's and Horace's advice is clear, it is uncertain how to interpret the \textit{exemplum} within the poem.  In addition to advice, Chiron also provides a prophecy---something that Horace cannot do.  But the prophecy itself is very grim.  Chiron makes clear that Achilles will die young.  Here are two suggestions for how to read the \textit{exemplum} back into the overall situation of the poem.  First, we are more likely to interpret the iambus as Horace and a companion nervous before a war since that is true of Achilles.  Second, the poem at least implies that life is likely to be short and death likely to be soon since that too was true for Achilles.  In these ways, the \textit{exemplum} emphasizes the larger \textit{carpe diem} theme of the poem.

The Centaurs were notoriously uncivilized and savage creatures, but Chiron was an exception.  While the other centaurs were born of a human father, Chiron's father was Cronus, the Titan and father of Zeus.  Ancient writers consistently portray Chiron as wise and just, again in distinct contrast to other centaurs.  Chiron appears in a number of myths as a tutor to young heroes, and he is often associated with medicine and healing.

Homer mentions Chiron four times in the \textit{Iliad}.  Early in the fighting, the doctor Machaon treats the wounded Menelaus using drugs that Chiron taught to Asclepius, Machaon's father (4.219).  Later we learn that Chiron taught Achilles the use of these same drugs and that Achilles in turn taught their use to Patroclus (11.832).  Twice Homer says that Chiron gave an especially heavy and fearsome ash-wood spear to Peleus, the father of Achilles (16.143 and 19.390).  Although Homer does not explicitly mention the idea that Chiron tutored Achilles more generally, these examples do suggest a special relationship between Chiron and the family of Achilles.\footnote{Homer tends to downplay or suppress magical or fantastical elements of traditional stories, especially in the \textit{Iliad}. Thus, Chiron's relationship with Achilles is far more limited in Homer than in other writers.  In the \textit{Iliad}, Homer even provides Achilles with a human tutor, Phoenix, whose existence rules out many of the stories of Achilles and Chiron.  For Homer's tendency to minimize magic in the \textit{Iliad}, see \citet[165 ff.]{griffin1983}.}

\comment{11}{11}

\lem{nobilis \lips alumno}: If we set aside the \textit{ut}---which Horace elegantly delays from the first position---this is a Golden Line with parallel word order.  The two adjectives \textit{nobilis} and \textit{grandi} precede the verb \textit{cecinit}, and the nouns \textit{Centaurus} and \textit{alumno} pick up the two adjectives in order.  The line has an additional element of \textit{doctrina}: Horace names neither Chiron nor Achilles.  Instead, he uses \textit{antonomasia}, the figure of speech where a writer substitutes an epithet or phrase for a name.\footnote{E.g. \textit{the Yankee Clipper} instead of Joe DiMaggio.} Readers must know the story in order to understand the references.\footnote{In mythology, Chiron tutors several heroes, so the phrase ``great fosterling'' does not necessarily refer to Achilles---though he is an obvious guess.  However, the following line settles the question with a reference to Thetis, Achilles' mother.}\indent\lem{grandi}: When used of people, this adjective means \textit{grown up, mature, full-grown} (\textbf{OLD} 1).  Thus, \citet{watson2003} glosses it as \textit{iam adulto}.  However, \citet{mankin1995} is surely right to see connotations of the word's meaning as applied to animals or inanimate objects: \textit{of considerable size, big, large, great} (\textbf{OLD} 2).\indent\lem{cecinit}: In addition to the meaning \textit{sing}, the verb \textit{cano} is strongly associated with prophecy in Latin.  Augustan poets like Vergil and Horace use words like \textit{vates}, which properly means \textit{seer}, to describe themselves as poets.  Thus, they elevate the status and importance of their writing.

\comment{12}{12}

\lem{invicte}: The initial placement of the vocative is emphatic and appropriate to high style.  Commentators have viewed this word as proleptic, referring to the future since they assume that Chiron is speaking to a young Achilles who is not yet ``undefeated''.  However, we might also take it more literally---in which case, it would have an ominous implication.  Up until now, Achilles has never met defeat, but Troy awaits, and there he will meet his end.\indent\lem{mortalis dea nate puer Thetide}: This appositive phrase expands on \textit{invicte}.  A whole line vocative phrase is again a feature of high style.  This clause is also another \textit{antonomasia}.  Horace never uses Achilles' name in the poem.\indent\lem{mortalis dea}: Horace emphasizes the vast gap between mortal son and divine mother by placing these two words next to each other.\indent\lem{dea \lips Thetide}: Ablative of source with \textit{nate}.  Thetis was a sea nymph, a daughter of Nereus, the so-called old man of the sea.   A prophecy predicted that her son would be ``greater than his father".  Zeus thus forced her to marry a human, Peleus, so that her child would be no threat to the Olympian order.  The wedding of Thetis and Peleus was the indirect cause for the Trojan war.  When Eris, goddess of strife, was not invited to the wedding, she threw an apple inscribed \textit{For the most beautiful} into a crowd of goddesses.  This led to the judgment of Paris: the Trojan Paris had to decide whether to give the apple to Hera, Athena, or Aphrodite.  He awarded the prize to Aphrodite, who in exchange helped him lure away Helen, the wife of Menelaus.  Menelaus and his brother Agamemnon gathered an enormous army from Greece and sailed to Troy to reclaim Helen and pay back the insult.\footnote{Much of the mythology concerning Thetis is post-Homeric and sub-epic.  For example, Homer refers only once---and in a brief and allusive manner---to the judgment of Paris (\textit{Iliad} 24.28-30).  However, that doesn't mean that these stories were \textit{invented} after Homer's time.  Some of them may have been, but many are clearly in the background of Homer's epics, even if he chooses not to mention them for reasons of decorum or suitability to the story he wishes to tell.  To consider the judgment of Paris again, even if Homer does not mention it, we are constantly reminded of it by the immense and otherwise unexplained hatred of Hera and Athena for Troy.  See for example \citet[88]{macleod1982} in his note on lines 25-30.}

\comment{13}{13}

\lem{manet} is ominous and threatening.\indent\lem{Assaraci tellus}: Another example of \textit{autonomasia}.  The land of Assaracus is Troy.  Assaracus is in the royal family but not the main line of Priam and Hector.  He is, however, a mythical anscestor of Aeneas.  Although Assaracus is not an especially well-known Trojan, the unobvious choice fits a \textit{doctus poeta} and it is also especially suitable for a Roman audience, as \citet[223]{mankin1995} observes.

\comment{13}{14}

\lem{quam \lips Simois}: The structure of this clause is as follows: \textit{quam\footnote{The relative refers to \textit{tellus}.} frigida flumina \dag parvi\dag Scamandri et lubricus Simois findunt.}  In the \textit{Iliad} the Scamander and Simois rivers are major landmarks on the plain of Troy where Greeks and Trojans battle every day.

\comment{13}{13}

\lem{parvi}: Easy to translate, but difficult to understand.  For further discussion see the chapter on textual criticism.

\comment{14}{14}

\lem{lubricus}: Although this adjective usually means \textit{gently flowing, gliding} when applied to rivers (\textbf{OLD} 3), most commentators have seen the epithet as more menacing: \textit{slippery, hazardous} (\textbf{OLD} 1, 4).\indent\lem{et} is delayed from the front of the clause to make way for \textit{lubricus}.

\comment{15}{16}

\citet[224--225]{mankin1995} comments that the Parcae were native Roman goddesses of birth, but over time the Romans associated them with the Greek Moirai or Fates.  They are represented here, as often, weaving out the destiny of mortals.

\comment{15}{15}

\lem{unde} must be taken closely with \textit{reditum} rather than with the clause's verb \textit{rupere}.  It refers back to Troy.\indent\lem{tibi} scanned \begin{metrica}t\-ib\=i rather than t\-ib\-i\end{metrica}, which is the default scansion.  This is a frequent metrical variant.\indent\lem{certo subtemine}: The \textit{subte(g)men} is the weft (or woof): transverse threads woven into a pattern under the lengthwise threads that form the warp.  (In Latin the warp is \textit{stamen}.)  The phrase is probably ablative of means, with \textit{rupere} though it could also be taken as a descriptive ablative with \textit{Parcae}.  The description \textit{certo} transfers the certainty of the Parcae to their instrument, the weft.  Achilles is \textit{certain} to die young, but Horace and his companions might (\textit{fortasse} 7) receive a change in luck.

\comment{16}{16}

\lem{rupere}: The perfect tense suits descriptions of fated events in Latin.  In \textit{Aeneid} Book 1, for example, Jupiter promises Venus that he has settled the fate of the Romans, declaring: \textit{imperium sine fine \textbf{dedi}} (1.279).\indent\lem{nec mater \lips}: Achilles will never return home, and in addition his mother will not carry back his corpse for burial.  These words increase the pathos of Chiron's prophecy, and they are another allusive hint that Horace and his companion may be facing the prospect of death far from home.\indent\lem{caerula}: Thetis, a water nymph, is the color of the sea.

\comment{17}{17}

\lem{illic} points forward to Troy.  Horace's advice is for the present (see, for example, \textit{nunc} in line 8), but Chiron's advice to Achilles is for later---when Achilles is in Troy.\indent\lem{omne malum vino cantuque levato}: The construction of \textit{levare} here is completely different from the example earlier in the poem on line 10.  In the earlier example, Horace urges that they ``relieve their hearts from harsh pains" (\textit{levare diris pectora sollicitudinibus}). The accusative is something good which must be emptied of an evil in the ablative.  In Chiron's words here, however, the accusative \textit{omne malum} is the evil, and \textit{vino} and \textit{cantu} are ablatives of means.  The similarity of the words binds the two speeches together, but it is a beautiful poetic touch that Horace varies the details of the phrasing.\indent\lem{levato} is a second person singular future active imperative.  The future imperative is rare.\footnote{See \textbf{NLG} \S 163b for forms and \S 449 for uses.}  It is grammatically appropriate for commands with a specific reference to the (less immediate) future.  However, by Horace's time, the future imperative had an archaic and grand tone, and Horace probably uses it here for that reason rather than because Chiron's advice will only be followed much later.

\comment{18}{18}

Take \textit{dulcis alloquiis} as ablative in apposition to \textit{vino} and \textit{cantu} above and \textit{deformis aegrimoniae} as objective genitive with \textit{alloquiis}.

\comment{18}{18}

\lem{deformis} means ugly or unsightly.  Here it is used metonymically with \textit{aegrimonia}, the misery which causes people to become ugly.\indent\lem{aegrimoniae} is a rare word for misery.  The root \textit{aeger} means \textit{wretched} or \textit{miserable}, and the suffix \textit{-monia} indicates a quality (or more rarely an activity).\indent\lem{alloquiis}: \textit{alloquium} is another rare word, made up of \textit{ad} and the root of the verb \textit{loquor}.  Literally it means any speech towards someone or something, i.e. an address.  But it often indicates, as here, comforting or consoling speech, encouragement.

This final line effectively sums up the poem as a whole: sweet comforts for disfiguring misery.  Horace never returns from the inset story of Chiron and Achilles to the frame.

% -]] Exemplum: Chiron's advice to Achilles 11-18
