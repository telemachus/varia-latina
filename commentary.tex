% [[- Chapter title
\chapter*{Commentary}
% -]] Chapter title

% [[- 1-2 Introduction addressed to posterity
\section*{1--2 To posterity}

Ovid addresses this poem to posterity, and he invites readers to learn his
life's story, so that they can better understand the kind of person he is.

\comment{1}{2}

The word order of the first couplete is difficult and elaborate.  Translate it
in this way: \textit{accipe, posteritas, ut noris qui fuerim ille ego, lusor
tenerorum amorum, quem legis}.  (If you're curious about how to arrive at this
word order, see the appendix on complex word order.)

\comment{1}{1}

\lem{Ille}: As Ovid will say later in the poem, he was extremely famous by the
time of his exile.  \textit{Ille} thus has its common meaning of `that famous'
or `that well-known'.\indent\lem{ego} refers to the speaking voice of these
poems.  In English we might ask ``What kind of person is that `I'?" to learn
about the character of the poet or narrator.\indent\lem{qui}: The interrogative
pronoun \textit{quis} asks about a name or identity, but when the interrogative
adjective is used as a substantive, as here, it asks about a person's nature or
character.  Compare the difference between the two questions ``Who was that?"
and ``What sort of person was he?"\indent\lem{lusor}: Catullus used the verb
\textit{ludere} to describe the playful composition of poetry, and by Ovid's
time it was almost a technical term for writing love poetry.  Hence, this
verbal noun means `writer' or `poet' here.

\comment{2}{2}

\lem{noris} is a syncopated form of the perfect subjunctive \textit{noveris}.
The tense of this verb is often chosen for semantics rather than time. Tenses
from the present stem indicate `learning; coming to know', while tenses from
the perfect stem indicate `having learned' and thus
`knowing'.\indent\lem{accipe}: This verb is very frequently used, as here,
without an explicit direct object to mean `hear, listen; attend to; learn'.
% -]] 1-2 Introduction addressed to posterity

% [[- 3-40 Ovid's origin, family, early life and education
\section*{3--40 Ovid's origin, family, early life and education}

Although this is not quite a traditional biography, Ovid covers many of the
genre's basic elements. He starts with his city of origin, telling us where his
family was from and their social status.  He also talks about his older
brother.  The two boys shared a birthday one year apart, and they had very
different temperments.  While his brother was a natural public speaker and
lawyer, Ovid gravitated towards literature.  Although his father tries to
convince him otherwise, Ovid gives up on a public career and pursues his
interest in poetry.

\comment{3}{3}

\lem{Sulmo} is a city in central southern Italy, in the modern Abruzzo region.
The Paeligni, an Italic tribe, founded and populated the city.  As early as
300 BCE, the Paeligni allied themselves with the Romans.  Ovid speaks of Sulmo
often in his poetry.  He took pride in his birthplace, although he also thought
of it as a very small town compared with Rome.

\comment{4}{4}

\lem{milia \lips noviens \lips decem}: Take these words together.  A thousand
paces is a Roman mile.  \textit{Nine times ten thousand} means ninety
miles.\indent\lem{urbe}: Ancient writers use \textit{urbs} to mean simply
`Rome', just as contemporary New Yorkers use \textit{the city} alone to mean
`Manhattan'.

\comment{5}{6}

\lem{nec non \lips cum}: No one would say or write \textit{and not not when} in
English, but such double negations are common in Latin writing.  Treat
\textit{nec non \lips cum} as a \textbf{litotes} with the force of \textit{just
in the year when}.

\comment{6}{6}

Ovid dates his birth by referring to a famous battle of that year.  In March of
43 BCE, the two consuls, Gaius Vibius Pansa and Aulus Hirtius, fought and
defeated Marc Antony at Mutina, in northern Italy.  However, both consuls
themselves died in the fighting.  Ovid allusively dates his birth to 43 by
describing it as the year \textit{when each consul fell}.

\comment{7}{8}

Ovid stresses that his family are not newly-made \textit{equites}, although he
also tries to give the impression that he doesn't care very much about such
distinctions (\textit{si quid id est}).

We must supply a main verb here in both clauses after \textit{si quid id est}.
The most straightforward thing is \textit{sum} alone for the first clause, and
then use that same \textit{sum} with \textit{factus} in the next line.

\comment{7}{7}

\lem{vetus ordinis heres}: The adjective \textit{vetus} is transferred and more
properly belongs with \textit{ordinis}.  In addition, we need to supply the
adjective \textit{equestris} to modify \textit{ordinis} as well.  Take the
genitive as descriptive.

\comment{8}{8}

\lem{modo} means `recently, now, just now', as often. It should be taken with
\textit{factus}.

\comment{9}{14}

Ovid had an older brother---we do not learn his name---who was born on the same
day as Ovid one year earlier.  This entire passage moves slowly from general to
specific: Ovid had a brother; they were born twelve months apart; they were
born on exactly the same day.

\comment{10}{10}

\lem{tribus ante quater mensibus} is more date-related math.  \textit{Three
months times four before} means `12 months before'.  That is, Ovid's brother
was born a year earlier.  By itself this does not mean \textit{exactly} one
year before, so Ovid makes that clear in the next couplet.

\comment{11}{11}

\lem{Lucifer}: Compound adjectives are rarer in Latin than in Greek, and they
were inherently grand and poetic.  This one comes from \textit{lux, lucis, f.}
and the suffix \textit{-fer}.  It is often used as a substantive referring to
the morning star, i.e. Venus, and that's what Ovid means here.  Take it
together with \textit{idem} at the end of the verse.\indent\lem{natalibus}: The
normal Latin for birthday is \textit{dies natalis}, and the phrase is common
enough that the noun can easily be supplied.

\comment{12}{12}

\lem{liba}: A \textit{libum} was a ceremonial cake or pancake made with flour,
milk, oil and honey.

\comment{13}{14}

In this last couplet, Ovid finally tells us the precise day that he and his
brother were born.  Again he moves from general to specific.  The first line
says only that their birthday was one of the five days of the
\textit{Quinquatrus}.  This was a multi-day festival for Athena held from March
\nth{19}--\nth{23}.  Ovid allusively tells us the exact day by saying that it
was the day `which was accustomed first to become bloody with battle'.  This
refers to gladiatorial games that took place during the \textit{Quinquatrus}.
These games began on the second day of the festival.  Hence, Ovid and his
brother were born on March \nth{20}.

\comment{16}{16}

\lem{imus}: The use of uncompounded forms of \textit{ire} is poetic.  A prose
author would use \textit{adimus} and then repeat the preposition \textit{ad} in
the sentence.\indent\lem{ab arte}: A variant for an ablative of respect with no
preposition.  Take with \textit{insignes}.

\comment{18}{18}

Extremely ornate word order: adjective-A adjective-B Central Word noun-a
noun-b.\indent\lem{verbosi}: Rome's Forum was central to the legal and
political life of the city.

\comment{19}{19}

\lem{caelestia sacra}: A grand periphrasis for poetry.  Ovid alludes to ideas
of poetry as divine and holy inspiration, though by his time such associations
are traditional commonplaces not literal beliefs.

\comment{21}{22}

Ovid's father tries to dissuade him from pursuing a career as a poet.  He
reminds Ovid that even Homer died poor.\indent\lem{Maeonides} is a patronymic,
i.e. a word that inherently means `son of'.  Patronymics are extremely common
in Homeric poetry.  This patronymic is somewhat unusual since the patronymic
suffix (\textit{-ides}) is attached to a place name rather than to the name of
a person.  Maeonia is an old name for Lydia, a famous city in Asia Minor.
Lydia itself was not one of the traditional cities that claimed Homer, but it
was in the Ionian region where many people believed Homer was born.  This type
of ``close enough" geographical reference is common in Roman poetry.

\comment{23}{24}

\lem{motus eram \lips temptabam}: Ovid frequently uses the pluperfect to
describe the background conditions for a narrative situation in the imperfect.

\comment{23}{23}

\lem{toto Helicone relicto}: An ablative absolute. Helicon is a mountain in
Boetia, traditionally thought of as the home of the Muses.

\comment{24}{24}

\lem{modis}: A separative ablative without any preposition, common in poetry.
Take closely with \textit{soluta}.

\comment{26}{26}

Supply \textit{id} or \textit{illud} as the antecedent of \textit{quod} and the
subject of \textit{erat}.\indent\lem{versus} is predicate nominative, not the
subject.

\comment{27}{27}

\lem{tacito passu}: Ablative of manner.\indent\lem{labentibus annis}: Ablative
absolute.

\comment{28}{28}

\lem{liberior \lips toga}: Refers to the \textit{toga virilis}, first worn by
Roman citizen males at roughly 16.  (Before that they wore the \textit{toga
praetexta}.)  This toga was `freer' insofar as it brought greater
independence.\indent\lem{fratri, mihi}: Datives of agent.  This construction
was initially limited to use with the passive periphrastic in Latin.  Over time
it began to be used with perfect passive participles as well.  Poets and later
prose writers used it as an alternative to the ablative of agent even more
widely---with any passive verb form.

\comment{29}{29}

\lem{cum lato \lips clavo}: A periphrasis for the adjective
\textit{laticlavius, -a, -um}, which cannot appear in a hexameter verse for
metrical reasons.  (Its first three syllables form a cretic.)  The adjective
itself means `with a broad purple stripe', and it refers to a social
distinction in Roman dress.  Senators, military tribunes of equestrian status
and some equites were allowed to wear a broad purple stripe on their togas.
The stripe marked their high political status.

\comment{30}{30}

\lem{studium} refers to Ovid's interest in poetry.

\comment{31}{40}

Ovid describes his brother's death briefly and without much explicit emotion,
but he also makes clear how great a blow it was to him.  In addition, the loss
of his brother (and his first political office) becomes a turning point in the
narrative.  After this point, he devotes himself completely to poetry.

\comment{31}{32}

More periphrastic time math: Ovid's brother ``had doubled ten years of life",
i.e. he was twenty.\indent\lem{cum perit}: A \textit{cum inversum}, that is
a sentence in which the main idea is placed in the gramatically subordinate
\textit{cum} clause and the temporal background is placed in the gramatically
independent clause.  Compare the English ``I was walking down the street
yesterday when suddenly I found a hundred dollar bill!"  The simplicity of
description belies and implicitly magnifies what his brother's death must have
meant to Ovid.\indent\lem{parte} is a separative ablative following
\textit{carere}.

\comment{33}{34}

A few years after his brother died, Ovid held his first political office.  He
would have been in his early twenties.  He was a member of one of the city's
groups of \textit{tresviri}.  There were a number of three-person offices: the
\textit{tresviri monetales} were in charge of making money; the
\textit{tresviri Capitales} were in charge of prisons; the \textit{tresviri
nocturni} were in charge of fire control; the \textit{tresviri epulones} were
in charge of providing public feasts.  We do not know which exact office Ovid
held.

\comment{35}{35}

\lem{curia restabat}: The Curia was the Roman Senate.  He had the choice to run
for higher office, but as the rest of the sentence makes clear, he
declined.\indent\lem{clavi mensura coacta est} is an allusive periphrasis.
Instead of saying that he decided not to seek senatorial offices, Ovid says
that the width of his stripe was narrowed.  Senators and their families, as
well as some equestrians, could wear the broad purple stripe mentioned on line
29.  But most equestrians wore only the \textit{clavus angustus} or `narrow
stripe'.  By changing his clothing, Ovid indicates that he abandoned his
father's political ambitions.

\comment{36}{36}

\lem{nostris viribus}: Ablative of comparison with \textit{maius}.  Be careful
to distinguish forms of \textit{vir, viri, m.}, a second declension noun, from
those of \textit{vis, --, f.}, an irregular third declension noun.  They may
look somewhat similar, but they have no exact forms in common.

\comment{38}{38}

\lem{sollicitae}: Like many adjectives formed from perfect passive participles,
this word has both an active and a passive meaning.  Passively it can mean
`anxious, disturbed', and actively it can mean `disturbing, distressing'.  Here
the active meaning makes more sense since Ovid found the pursuit of political
office positively stressful.\indent\lem{ambitionis}: Objective genitive
following \textit{fugax}.

\comment{39}{39}

\lem{petere \lips suadebant}: Supply \textit{me} as subject of the
infinitive.\indent\lem{Aoniae \lips sorores}: Aonia is a geographical term
related to Boetia.  Hence, the `Aonian sisters' are the Muses, who live on Mt.
Helicon in Boetia.
% -]] 3-40 Ovid's origin, family, early life and education

% [[- 41-68 Ovid and poetry
\section*{41--68 Ovid and poetry}

Ovid gives a brief and allusive but very suggestive portrait of the Roman
poetic world of the 20s BCE.  He also discusses his own \textit{Amores}, his
first serious publication.  By a careful transition, Ovid also introduces the
subject of his exile for the first time at the end of this section.

\comment{41}{41}

\lem{temporis illius} is a descriptive genitive, and \textit{illius} again has
the force of `that famous'.

\comment{42}{42}

\lem{vates}: Originally the word \textit{vates, vatis, m.} meant a `prophet' or
`seer'.  The Augustan poets began to use the word as a elevated synonym for
`poet'.\indent\lem{rebar adesse deos}: The previous clause---\textit{quot
aderant vates}---is equivalent to \textit{omnes vates}, and it provides the
implicit subject accusative of \textit{adesse}.  The word \textit{deos} is
predicate accusative in the indirect statement.

\comment{43}{44}

Ovid's older contemporary, Aemilius Macer, was a didactic poet from Verona,
like Catullus.  Ovid alludes to three of his poems here: a poem on birds,
a second on snakes, and a third on plants.  Macer died around 16 BCE.

\comment{45}{46}

Sextus Propertius was an elegiac poet from Assisium in northern central Italy.
He wrote four books of elegies.  The first three focus primarily on his
tortured and violent relationship with Cynthia, but in the fourth book he turns
from love to Roman myths and history.  He was born circa 50 BCE and died
sometime after 15 BCE.

\comment{46}{46}

\lem{qui} is delayed from the beginning of the verse to make way for the more
meaningful initial phrase \textit{iure sodalicii}.  Hellenistic Greek poets
cultivated this stylistic device, and Roman poets imitated and continued it.

\comment{47}{48}

Ponticus was an epic poet about whom scholars know very little. Propertius
dedicated two poems (I 7 and I 9) to him.  Propertius I 7 suggests that he
wrote an epic about the Theban cycle of myths.  Bassus was an iambic poet, and
scholars know even less about him.  Propertius I 4 is addressed to a Bassus,
and they may be the same person.

\lem{Ponticus heroo}: Supply \textit{versu} to go with and apply
\textit{clarus} from the next phrase. Ponticus and Bassus are subjects of
\textit{fuere} in the next line, and \textit{dulcia membra} is the
predicate.\indent\lem{membra}: \textit{membrum} can be used of people---the
members or parts of a group or organization---as well as of limbs or body
parts.\indent\lem{convictus} is a form of the noun \textit{convictus, -us, m.}
not a participle.

\comment{49}{50}

Quintus Horatius Flaccus, 65 BCE--8 CE, was born in Venusia in southern Italy.
He claims to have come from a very humble family, but his level of education
suggests that he might have exaggerated.  He wrote poems in a variety of
genres---satire, lyric and epistolary poetry.  But he's most famous for his
\textit{Odes}, lyric poems adapting the meters and themes of famous Greek
poets.

Macer and Propertius frequently recited their poems to Ovid (\textit{saepe} 43,
45), and he claims close association (\textit{convictus} 48) with Ponticus and
Bassus.  By contrast, Horace--an older poet and already quite famous by the
20s---Ovid only heard.

\comment{49}{49}

\lem{numerosus}: Horace was deservedly proud of his lyric poems.  No other
Roman poet, before or after, handled as many complex Greek meters with as much
skill in Latin.

\comment{50}{50}

\lem{ferit Ausonia carmina culta lyra}: A poetic inversion of a standard
musical expression.  Normally one would `strike a lyre with a plectrum'
(\textit{ferire lyram plectro}), but here Ovid describes Horace as `striking
his songs with a lyre'.  Scan this line carefully to match adjectives and nouns
properly.

\comment{51}{51}

\lem{tantum}: Adverbial meaning `only', not an adjective here.

\comment{51}{52}

Albius Tibullus, circa 55--19 BCE, was an elegiac poet who wrote two books of
poems.  The first book is addressed primarily to a woman he calls Delia and the
second to a woman he calls Nemesis.  Tibullus is unusual for an Augustan writer
since he never refers to Augustus in his poems.

\lem{nec \lips meae}: Instead of saying that he didn't get enough time to
develop a friendship with Tibullus, Ovid says that fate didn't give
\textit{Tibullus} enough time to become Ovid's friend.  This probably isn't
arrogance so much as a desire to avoid easily predictable phrasing.

\comment{53}{54}

Gaius Cornelius Gallus, circa 70--26 BCE, was a Roman politician and poet from
Forum Livii in Northern Italy.  After quelling a revolt in Egypt in 29 BCE,
Gallus apparently took too much personal credit for his victory.  Augustus
recalled him, and Gallus later took his own life.  As a poet, Gallus is most
famous as the creator of the specifically Roman style of elegy: highly
emotional poems to one beloved.  Gallus wrote to a woman that he called
Lycoris, but we have nearly nothing left of his poems.  So it's hard to judge
him as a writer.

Ovid provides a brief succession of elegiac poetry at Rome, culminating in
himself.  Ovid says that Gallus was the first elegist, then Tibullus,
Propertius and finally Ovid.  After Ovid no other Roman wrote complete books of
Roman elegies, although a number of important writers composed individual
elegiac poems.  But as a genre of its own, elegy ends in Rome with Ovid, though
he wouldn't have known that when he wrote.

\comment{55}{55}

\lem{utque ego maiores}: Supply \textit{colui} from \textit{coluere} in the
next clause.

\comment{56}{56}

\lem{non tarde} is a litotes for \textit{valde celeriter}.\indent\lem{Thalia
mea}: Thalia is one of the Muses.  Ovid uses the phrase \textit{my Thalia} here
to mean `my poetry'.  Ovid may have chosen Thalia because she was particularly
associated with comedy and light poetry.

\comment{57}{60}

Whatever poems Ovid has in mind in lines 57--58, they aren't extant.  He very
likely never published them more formally.  Ovid mentions his juvenalia here
only to emphasize how young he was when he published earliest serious poetry,
his \textit{Amores}.

\comment{58}{58}

\lem{resecta \lips fuit}: Roman writers sometimes use \textit{super}-pluperfect
forms like this, a participle combined with the perfect tense, instead of the
more common particple combined with the imperfect.  There doesn't seem to be
a difference in meaning.\indent\lem{bisve semelve}: Another poetic inversion of
the expected order.

\comment{59}{60}

Corinna is the name Ovid gave to the woman he wrote most of his \textit{Amores}
to.  Scholars cannot decide whether Corinna was a genuine person or a character
Ovid created in order to have a reason to write love poetry.

The bare bones of the sentence is \textit{Corinna ingenium moverat}.  However,
Ovid modifies the subject \textit{Corinna} with two participial phrases:
\textit{cantata per totam urbem} and \textit{dicta nomine non vero}.

\comment{60}{60}

\lem{non vero}: Ovid uses this instead of \textit{falsum} for the sake of
variety.\indent\lem{mihi}: The most likely interpretation of this is as
a dative of agent with \textit{dicta}, but in theory we might also share it
as a dative of reference with \textit{moverat ingenium}.

\comment{61}{64}

Ovid tells us that he wrote a great deal, but he burned what didn't meet his
standards.  But when he was forced to flee into exile, he burned even poems he
thought were good out of anger.  This is the first mention of exile in the
poem.  Ovid quickly moves away from that topic, but he leverages 

Supply \textit{ea} to be the antecedent of \textit{quae} and the direct object
of \textit{dedi}.\indent\lem{emendaturis}: The future active participle
expresses purpose here, a poetic usage that eventually makes it into later
Latin prose authors as well.
% -]] 41-68 Ovid and poetry
