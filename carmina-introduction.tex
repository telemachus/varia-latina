% [[- Chapter title
\chapter*{Introduction}
% -]] Chapter title

% [[- Horace's life
\section*{Horace's Life}

Horace wrote a great deal about his own life and background, but unfortunately this does not mean that we know a great deal about him.  Ancient writers, especially the poets, often use their own character and history for rhetorical effect.  As an example, Horace writes that he was a coward at the battle of Philippi (C II 7.9-10): that he threw down his shield and fled. Is this the honest admission of an older man? Perhaps, but we must be cautious.  It suits Horace to say that he behaved thus since he is living under the rule of one of the victors at Philippi (Augustus), and Horace fought on the losing side.  In addition several Greek lyric poets claimed to have thrown away their shields and fled from a battle---exactly as Horace says he did in this lyric poem.  So his admission may also be a nod to the genre of poetry he's writing.  This small example should already show how difficult it is to separate fact from fiction in Horace's autobiographical statements.  Nevertheless, I will present the broad strokes of the standard biography of Horace and then say a little about why I think readers should be cautious.

Quintus Horatius Flaccus was born on December 8\textsuperscript{th}, 65 BCE in Apulia in southeastern Italy.  His father was a former slave who owned a small farm and worked as an auctioneer to make additional money.  Instead of sending Horace to the local school, he brought the young Horace to Rome and made sure he received an excellent education.  After Rome Horace went to Athens to continue his studies.  He was there when Brutus arrived in Athens after the murder of Caesar (March 15\textsuperscript{th}, 44 BCE).  Brutus was recruiting young Romans for his army, and Horace joined him.  Brutus made Horace a military tribune, and Horace was present at the battle of Philippi (October 42 BCE) where the pro-Caesarian forces of Anthony and Octavian defeated the army of Brutus and Cassius.  Horace fled the battle and survived.  Eventually he made his way back to Italy.  He had lost his family's property as a result of proscriptions during the civil war, but he had enough money to purchase an archival position in the government, the office of \textit{scriba quaestorius}.\footnote{In Rome, some of what we would call civil service jobs could be \textit{purchased}.}  He began (or continued?) to write poetry, and eventually his skill brought him to the attention of literary figures such as Vergil.  With the help of these supportive readers, Horace came under the patronage of Maecenas, an aide to Augustus, and eventually Augustus himself.  Horace enjoyed a remarkable literary career, and he served as a kind of poet laureate or national poet when Augustus asked him to write a hymn---the \textit{Carmen Saeculare} for a historic national festival held in 17 BCE.  He died on November 27\textsuperscript{th} 8 BCE.

I think that we should be suspicious of the overall arc of this story: a poor country boy, raised by a caring and virtuous father, finds success in the big city through hard work and good character.  In particular, I follow Gordon Williams in thinking that Horace's father was not an ex-slave \citep{williams1995}.  No single part of the traditional biography is impossible, but the accumulation of unlikely combinations troubles me.  Horace's father was poor, but well-off enough to have his son educated in the manner of an \textit{eques} or senator.  He was a modest farmer and businessman from the country, but he had the social connections to find and enroll his son at an elite school in Rome.  There was money enough for Horace to travel abroad and study further in Athens.  Although Horace's family had servile origins, Brutus nevertheless made Horace, who had no previous military experience, a tribune in his army.  After fighting on the wrong side at Philippi and losing his family's property to the chaos of the civil war, Horace still had the connections and money to purchase the office of \textit{scriba quaestorius}.\footnote{\citet[104]{williams1995} notes that Horace's claims of poverty belie his being ``sufficiently wealthy to be able to purchase the highly profitable post".}

Rather than take the traditional biography at face value, I would argue that it is Horace's chosen \textit{persona}.  In addition to the various \textit{personae} that Horace adopts as a character in his work, he also carefully cultivates a meta-\textit{persona} as Author.\footnote{For a near-contemporary parallel to the distinction between character and author, compare Caesar's \textit{Bellum Gallicum}.  The historical agent is always described in the third person as `Caesar', but the author of the work frequently uses the first-person to refer to other parts of the narrative.}  Horace presents the writer `Horace' as exactly the sort of hard-won, virtuous success story of the traditional biography.

% -]] Horace's life

% [[- Horace's Carmina and the two spring poems

% -]] Horace's Carmina and the two spring poems

% [[- L. Sestius P. f. L.

% -]] L. Sestius P. f. L.

% [[- Meter
\section*{Meter}

The meter of \textit{Carmen} I 4 is the third Archilochian.\footnote{This is the only example of this meter in Horace's lyric poems.} This meter uses couplets made up of (1) a greater Archilochian followed by (2) an iambic trimeter catalectic.  A greater Archilochian is a complex line.  It contains a dactylic tetrameter combined with an ithyphallic.  An iambic trimeter is simply three iambic feet, and a catalectic verse lacks a syllable or foot at the end of the line. In the case of an iambic trimeter catalectic, the line has eleven instead of twelve syllables.\newline

\indent\metra{\m\mbb\m\mbb\m\mbb\m\bb\c\m\b\m\b\m\m}

\indent\indent\metra{\bm\m\b\m\m\c\m\b\m\b\m\m}\newline
% -]] Meter

% [[- About The Text
\section*{About The Text}

I made the text for this edition by comparing the editions of \citet{sb1985} and \citet{klingner1959}. Everything in my text comes from one of those two editions. I also drew alternative readings from their notes.

Below the text is an \textit{apparatus criticus} that gives information about difficulties in the text.  The apparatus is minimal and follows the style used in several recent volumes in the \textit{Cambridge Greek and Latin Classics} series.  I use what is called a `positive' apparatus.  Every note begins with the reading I adopt and where it comes from.  After that I list alternatives of interest.  

The key below explains how the apparatus presents this information.

\begin{description}%
    [style=sameline,leftmargin=70pt,labelwidth=\widthof{\textbf{Name}}]
    \item[m] one or more manuscripts
    \item[M] the consensus opinion of most or all of the manuscripts
\end{description}

For further discussion of textual issues, see the chapter on Textual Notes later in the book.

% -]] About The Text
