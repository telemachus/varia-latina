% [[- Chapter title
\chapter*{Commentary}
% -]] Chapter title

% [[- Arrival of spring
\section*{Arrival of spring 1--8}

Horace begins by describing the arrival of spring in two stanzas.  The first describes the changes in weather and the effect of those changes on the human world.  The second stanza depicts spring's effect at the divine level.

\comment{1}{4}

The first stanza is a single, carefully-constructed sentence.  Horace uses an elaborate ascending tricolon to organize the opening: \textbf{(1) Solvitur \lips (2) trahuntque \lips (3) ac (a) neque \lips (b) nec}.  In addition, the stanza is shaped around a chiasmus: the first and fourth lines focus on the weather and the natural world; the second and third depict human actions and reactions to the change in weather. 

\comment{1}{1}

\lem{Solvitur}: On the one hand, it is the earth that is released from winter's grip (cf. \textit{terrae \lips solutae} on line 10).  On the other hand, the snow and frost itself is literally dissolved as it melts, releasing land and sea. Cf. \textit{dissolve frigus} (\textit{C} I 9.5).\indent\lem{acris}: Cold is traditionally \textit{sharp, piercing, penetrating}.  Ennius and Lucretius, two earlier poets, also described \textit{hiems} with \textit{acer, acris, acre}. See also Horace's own \textit{gelu \lips acuto} (\textit{C} I 9.3--4).\indent\lem{veris et Favoni}: Nearly a hendiadys: \textit{spring which the west wind announces}.  (The genitives are likely subjective with the verbal idea in \textit{vice}.)\indent\lem{Favoni}: The Favonius, or the west wind, is a standard symbol of the arrival of spring and good weather.  Roman poets often use the Greek name for the west wind, \textit{Zephyrus}, rather than the native Latin \textit{Favonius}.  This is the first of several distinctly Roman touches to the poem.
% -]] Arrival of spring
