% [[- Chapter title
\chapter*{Commentary}
% -]] Chapter title

% [[- Cynthia prima
\section*{Cynthia ruined me 1--1}

According to Propertius, Cynthia was the first person to create in him the overwhelming, ruinous desire that he suffers from.

\comment{1}{1}

\lem{Cynthia}: Propertius follows the tradition set by earlier poets, or so we are told, by using a Greek name for his beloved.  According to ancient tradition, her real name was Hostia, but we have no significant information about her biography.  Propertius emphatically sets her name first in this poem, and she will dominate all of Book I of his \textit{Elegiae}.\indent\lem{miserum}: This adjective is virtually a technical term in Roman love poetry.  Anyone who loves is, by definition, \textit{miser} or \textit{misera}.\indent\lem{cepit}: Suggests either a military or servile context: Cynthia has overwhelmed Propertius forcefully. Again, following Catullus, he portrays himself in a submissive role vis-a-vis his lover.\indent\lem{ocellis}: The placement after \textit{cepit} and the the diminutive run contrary to expectation: Cynthia conquered Propertius \textit{with her little eyes}.  (Keep in mind the various possible connotations of a diminutive in Latin.)

\comment{2}{2}

\lem{contactum}: This verb suggests moral or physical corruption.\indent\lem{ante} functions here quasi-adjectively, describing \textit{cupidinibus}.  I.e. ``previous desires".

\comment{3}{4}

\lem{deiecit \lips et \lips pressit Amor}: The subject is delayed, leaving readers to assume that Cynthia is still subject.  The military metaphors continue in these lines.

\comment{3}{3}

\lem{constantis \lips fastus} is genitive of description with \textit{lumina}.  Propertius once was proud, but now he is forced to keep his eyes lowered.  Again this suggests a subordinate status for Propertius.\indent\lem{lumina}: The word \textit{lumen} is often used in Latin poetry to mean \textit{oculus}.

\comment{4}{4}

\lem{impositis pressit Amor pedibus}: This striking and highly visual image perhaps reflects a painting or statue, though not necessarily one with Amor as its subject.  Amor is not normally represented in so aggressive and militaristic a fashion in ancient art.

\comment{5}{6}

These lines recall Catullus in spirit and language.  Propertius \textit{hates} (\textit{odisse}) pure, proper women, exactly the sort of women he should desire, and he \textit{lives with no plan} (\textit{nullo vivere consilio}), exactly the opposite of a Roman ideal.  Although Amor is now the grammatical subject, the blame is still in a larger sense Cynthia's, according to Propertius' logic.  She has ruined him.

\comment{5}{5}

\lem{castas}: A strong and, in any normal circumstance, very positive term: morally pure, virtuous.  It is almost a contradiction in terms to say that one \textit{hates} such a person.

\comment{6}{6}

\lem{improbus}: Emphatic since delayed significantly after the verb it modifies (\textit{docuit}) and placed first in its line.  The use of the nominative adjective instead of an adverb is idiomatic Latin.  Although the adjective is always negative in meaning, its tone varies a great deal.  It can be serious and angry (\textit{wicked, vile, base, shameless}) or said with almost a wink and a smile (\textit{mischievous, terrible, awful}).  What tone do you think it has here?\indent\lem{nullo \lips consilio}: Ablative of manner and a virtual litotes. To live \textit{with no plan} is to be \textit{completely out of control}.

\comment{7}{8}

Propertius has been suffering in this way for a year.

\comment{7}{7}

\lem{mihi}: The dative is felt with both \textit{furor} (as a possessive dative) and \textit{deficit} (as a dative of reference).  The pronoun is also emphatic: Propertius still feels the same, but Cynthia may not.\indent\lem{iam toto \lips anno}: Ablative where we would expect an accusative of duration of time.  In colloquial Latin, the ablative sometimes appears with the same meaning as the accusative of duration of time.\indent\lem{deficit}: The use of a present tense to describe an ongoing action that began in the past is common in Latin (and still now in modern Romance languages).  In English, we would tend to say something more like ``for the whole year, this feeling hasn't left me".

\comment{8}{8}

\lem{adversos \lips habere deos}: The phrase \textit{to have hostile gods} is elliptical.  It suggests that however much Propertius might pray for a happy relationship with Cynthia, the gods are set against him and will not grant him his wish.  Thus the couplet as a whole stresses that the raging desire (\textit{furor}) that he has for Cynthia is just as strong despite how bad the relationship is going.
% -]] Cynthia prima
