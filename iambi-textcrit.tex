% [[- Chapter title
\chapter*{Textual Notes}
% -]] Chapter title

% [[- O friend, friends, as friends, or Friend?
\section*{What kind of friend?}

\comment{3}{3}

\lem{amici}: This textual crux is famous, and undoubtedly the most important for our understanding of the poem.  The manuscripts universally read \textit{amici}.  The most natural way to take that would be as a plural vocative---this suits \textit{rapiamus} perfectly.  However, in line 6 Horace addresses \textit{tu}: a single, unnamed person.  Many readers have found the switch between plural \textit{friends} and a singular \textit{you} jarring.  In addition, this leaves the poem without a named addressee---very unusual for a poem of this kind.  To solve the first problem, some readers take \textit{amici} as nominative, in which case it would modify the understood subject of \textit{rapiamus} and it would function quasi-adverbially: let us grab, as friends, the opportunity from the day.  I find it hard to read the lines that way, and this suggestion doesn't solve the latter problem.  The poem still lacks an addressee.

A. E. Housman suggested \textit{Amici}: the singular vocative of the name \textit{Amicius}.  Who better to share consoling wine and song with than a friend named Friend?  However, this is a very uncommon Latin name, and it is likely to confuse readers---at least initially.  Thus, some scholars have suggested other names.  For example, C. O. Brink proposes \textit{Apici} from \textit{Apicius}.

\citet[205]{gaskin2013} complains that Housman's emendation is ``too clever by half".  He argues that without punctuation and capitalization, no ancient reader could read \textit{RAPIAMUSAMICI} and \textit{not} see it as a vocative (or possibly nominative) plural, addressed to Horace's fellow drinkers.  On the other hand, \citet[417, note 14]{lowrie1992} argues with equal conviction that ``the ambiguity involved is no more than transient".  As we read further into the poem, we naturally revise our initial understanding of \textit{amici} to \textit{Amici}.  Lowrie notes that a number of Horace's poems are addressed to people whose names are suspiciously appropriate to the message of the poem.  For example,  a drinking poem is addressed to Thaliarchus, a Greek name that means \textit{leader of the party} (\textit{C} I 9).  Or a poem about the inevitability of death is addressed to a Postumus (\textit{C} II 14).\footnote{As an adjective \textit{postumus, -a, -um} means `last born', but it is particularly used in a legal context to describe a child born after a father's death (\textbf{OLD} 1b).}

These arguments create a stalemate.  I'm not convinced that Housman's emendation \textit{cannot} be right, but I'm not convinced that it's right either.  As a result, I've left \textit{amici} in the text.  I'm unhappy with a sympotic poem without a named addressee, but the poem can make sense as it is.

For the sake of completeness, I'll mention a few other emendations.  Richard Bentley suggests reading \textit{amice}.  This would remove the startling shift from a plural to singular addressee, but the poem then has no named addressee.  The improvement doesn't seem great enough to overrule the universal manuscript reading.  Arthur Palmer proposes \textit{amica}.  This has the same problems as Bentley's emendation, and it adds a potential romantic element that seems out of place in this poem.

% -]] O friend, friends, as friends, or Friend?

% [[- diris or duris?
\section*{Dire or difficult anxieties?}

\comment{10}{10}

\lem{diris}: The main manuscripts agree on \textit{diris} from \textit{dirus, dira, dirum}.  Following Richard Bentley, \citet{sb1985}  prints \textit{duris} which has the support of some later copies of the poem.  He doesn't say why he prefers one reading over the other; it may be that he believes that \textit{dirus} is too strong a word for \textit{sollicitudinibus}.  However, both \citet[220]{mankin1995} and \citet[430]{watson2003} make good sense of the passage with \textit{diris}.  The emendation thus seems unnecessary.

% -]] diris or duris?

% [[- Scamander parvus?
\section*{A small Scamander?}

\comment{13}{13}

\lem{flavi}: Although the manuscripts universally offer \textit{parvi}, \citet{sb1985} prints \textit{\dag parvi\dag}.  This symbol---called a \textit{crux desperationis} or \textit{obelus}---is an editor's choice of last resort.  It indicates that the editor believes (1) that the text given by the manuscripts cannot possibly be what the original author wrote, but also (2) that no emendation is good enough to print instead.

The problem with \textit{parvi} is that Homer stresses the great size and strength of the Scamander.  In particular Homer describes the Scamander as μέγας ποταμὸς βαθυδίνης, a `large, deep-swirling river' (\textit{Iliad} 20.73).  The Scamander is fearsome rather than small.  In a famous scene from the \textit{Iliad}, the Scamander rises up in rage and nearly drowns Achilles after Achilles clogs the river with Trojan corpses.  Hephaestus, the god of fire, must step in to save Achilles from drowning (\textit{Iliad} 21.211-382).  Given this literary background, it's difficult to see why Horace would ever call the Scamander `small'.  However, for a defense of the manuscript reading see \citet[223--224]{mankin1995} and \citet[207--209]{gaskin2013}.

An ancient commentator offers \textit{pravi} from \textit{pravus - crooked, twisted}.  \citet[433--434]{watson2003} adopts this reading and offers a defense of it, taking the meaning here to be \textit{winding}.\footnote{Watson also interprets the Simois' epithet (\textit{lubricus}) in the same way as \textit{winding, tortuous} (434).  It seems very unlikely to me that Horace would describe both rivers in the same way so close together.}

Several emendations have been proposed for this crux. Richard Bentley suggested \textit{proni}, which would mean \textit{downward flowing} and by implication \textit{rushing}. Nikolaes Heinsius proposed \textit{flavi}, \textit{yellow}, \textit{golden}. This would be a learned reference to the \textit{Iliad}. Homer writes that the Scamander has two names: humans call the river \textit{Scamander}, but the gods call it \textit{Xanthus} (\textit{Iliad} 20.74). The adjective ξανθός (\textit{xanthos}) is equivalent in meaning and use to the Latin \textit{flavus}, so  Horace would refer to both of the river's names in one description by calling it \textit{Scamander flavus}. Peter Peerlkamp emended to \textit{puri}, meaning \textit{pure} or \textit{clear}.

% -]] Scamander parvus?

% [[- Certain or short?
\section*{A certain or a short thread?}

\comment{15}{15}

\lem{certo subtemine}: If we keep the manuscript reading \textit{certo}, then the certainty is transferred from the action of the Parcae to the weft.  On the other hand, Richard Bentley suggests \textit{curto}.  This also makes good sense: Achilles will not return home because of the ``short thread" of his fate.

% -]] Certain or short?
