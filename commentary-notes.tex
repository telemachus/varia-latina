\lem{de die}: The meaning of the preposition here is not entirely obvious.  With a period of time, \textit{de} often means \textit{starting with, at} (\textbf{OLD} 3); and \textit{de die} with a reference to drinking often refers to day-drinking, which Romans considered extravagant and potentially scandalous.  In addition, \textit{de} can indicate the person or thing from whom something is taken or received, as well as the person over whom a victory is won (\textbf{OLD} 6).  The primary meaning is probably \textit{from the day} (as source), but the suggestions of early drinking and victory over the bad weather are also likely present.

\lem{tu}: As \citet[136]{kilpatrick1970} explains, the guest would request a specific wine from the host at a Roman party.  So the addressee here is likely to be the person throwing the party, though we don't know who that is.

Remove bit on \textit{move} about coming down.

\textit{certo} versus \textit{fortasse}

\textit{nec mater domum} echoes Iliad 18.59--60. Thetis is speaking: ``I will not welcome Achilles in the future as he returns home to the house of Peleus." 
