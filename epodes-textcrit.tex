% [[- Chapter title
\chapter*{Textual Notes}
% -]] Chapter title

% [[- O friend, friends, as friends, or Friend?
\section*{What kind of friend?}

\comment{3}{3}

\lem{Amici}: This textual crux is famous, and undoubtedly the most important for our understanding of the poem.  The manuscripts universally read \textit{amici}.  The most natural way to take that would be as a plural vocative---this suits \textit{rapiamus} perfectly.  However, in line 6 Horace addresses \textit{tu}: a single, unnamed person.  Many readers have found the switch between plural \textit{friends} and a singular \textit{you} jarring.  In addition, this leaves the poem without a named addressee---very unusual for a poem of this kind.  To solve the first problem, some readers take \textit{amici} as nominative, in which case it would modify the understood subject of \textit{rapiamus} and it would function quasi-adverbially: let us grab, as friends, the opportunity from the day.  I find it hard to read the lines that way, and this suggestion doesn't solve the latter problem.  The poem still lacks an addressee.  So instead, I've printed an emendation suggested by A. E. Housman.  He treated \textit{Amici} as the \textit{singular} vocative of the name \textit{Amicius}.  This is not the most common Latin name, and it is likely to confuse readers---at least initially.\footnote{See below for more about this confusion.}  Thus, some scholars have suggested other names.  (For example \textit{Apici} from \textit{Apicius}.)  However, Horace shows a fondness for speaking names, and this poem suits an addressee whose name is essentially Friend.\footnote{\citet[205]{gaskin2013} complains that Housman's emendation is ``too clever by half".  He argues that without punctuation and capitalization, no ancient reader could read \textit{RAPIAMUSAMICI} and \textit{not} see it as a vocative (or possibly nominative) plural, addressed to Horace's fellow drinkers.  On the other hand, \citet[417, note 14]{lowrie1992} argues with equal conviction that ``the ambiguity involved is no more than transient".  As we read further into the poem, we naturally revise our initial understanding of \textit{amici} to \textit{Amici}.  Lowrie also relies on Horace's fondness for names that fit the theme of the poem to help her case.  I'm not sure that we can solve the subjective question of how a Roman reader would have interpreted the poem, but I would add the following to Lowrie's argument: If we assume the existence of a real person named Amicius, the problem is significantly smaller.}

For the sake of completeness, I'll mention a few other emendations.  Richard Bentley suggests reading \textit{amice}.  This would remove the startling shift from a plural to singular addressee, but the poem then has no named addressee.  Arthur Palmer proposes \textit{amica}.  This has the same problems as Bentley's emendation, and it adds a potential romantic element that seems out of place in this poem.

% -]] O friend, friends, as friends, or Friend?

% [[- duris or diris?
\section*{Difficult or dire anxieties?}

\comment{10}{10}

\lem{duris}: The main manuscripts agree on \textit{diris} from \textit{dirus, dira, dirum}.  Following Richard Bentley, \citet{sb1985}  prints \textit{duris} which has the support of some later copies of the poem.  He doesn't say why he prefers one reading over the other, but to my mind \textit{diris} is too strong a word for \textit{sollicitudinibus}.  \citet[220]{mankin1995} offers a defense of the manuscripts' reading.

% -]] duris or diris?

% [[- Scamander parvus?
\section*{A small Scamander?}

\comment{13}{13}

\lem{flavi}: Although the manuscripts universally offer \textit{parvi} here, I've printed Nikolaes Heinsius' emendation instead.  The problem with \textit{parvi} is that Homer always stresses the great size and strength of the Scamander.  In particular Homer describes the Scamander as μέγας ποταμὸς βαθυδίνης, a `large, deep-swirling river' (\textit{Iliad} 20.73).  The Scamander is also a fearsome river.  In an extremely famous scene, the Scamander rises up in rage and nearly drowns Achilles after Achilles clogs the river with Trojan corpses.  Hephaestus, the god of fire, must step in to save Achilles from drowning (\textit{Iliad} 21.211-382).  Given this literary background, it's difficult to see why Horace would ever call the Scamander `small'.  However, for a defense of the manuscript reading see \citet[223--224]{mankin1995} and \citet[207--209]{gaskin2013}.

Several emendations have been proposed for this crux.  Richard Bentley emended to \textit{proni}---which would mean \textit{downward flowing} and by implication \textit{rushing}.  Peter Peerlkamp emended to \textit{puri}, \textit{pure} or \textit{clear}.

Finally, \citet{sb1985} prints \textit{\dag parvi\dag}.  This symbol---called a \textit{crux desperationis} or \textit{obelus}---is an editor's choice of last resort.  It indicates that the editor believes (1) that the text given by the manuscripts cannot possibly be what the original author wrote, but also (2) that no emendation is good enough to print instead.

% -]] Scamander parvus?

% [[- Certain or short?
\section*{A certain or a short thread?}

\comment{15}{15}

\lem{certo subtemine}: If we keep the manuscript reading \textit{certo}, then the certainty is transferred from the action of the Parcae to the weft.  On the other hand, Richard Bentley suggests \textit{curto}.  This also makes good sense: Achilles will not return home because of the ``short thread" of his fate.

% -]] Certain or short?
